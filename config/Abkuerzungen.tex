% Akronyme
\newacronym{HTML}{HTML}{HyperText Markup Language}
\newacronym{FHDW}{FHDW}{Fachhochschule der Wirtschaft}
\newacronym{URL}{URL}{Uniform Resource Locator}
\newacronym{CSS}{CSS}{Cascading Style Sheets}
\newacronym{SSI}{SSI}{Server-Side Includes}
\newacronym{DOM}{DOM}{Document Object Model}
\newacronym{API}{API}{Application Programming Interface}
\newacronym{KPI}{KPI}{Key Performance Indicator}
\newacronym{MFE1}{MFE1}{Microfrontend 1}
\newacronym{MFE2}{MFE2}{Microfrontend 2}
\newacronym{MFE3}{MFE3}{Microfrontend 3}
\newacronym{MFE4}{MFE4}{Microfrontend 4}
\newacronym{ID}{ID}{Identifier}
\newacronym{UX}{UX}{User Experience}
\newacronym{UI}{UI}{User Interface}
\newacronym{DDD}{DDD}{Domain-Driven Design}
\newacronym{CDN}{CDN}{Content Delivery Network}
\newacronym{SPA}{SPA}{Single Page Application}
\newacronym{CORS}{CORS}{Cross-Origin Resource Sharing}
\newacronym{OIDC}{OIDC}{OpenID Connect}
\newacronym{SAML}{SAML}{Security Assertion Markup Language}
\newacronym{CI/CD}{CI/CD}{Continuous Integration, Continuous Delivery und Continuous Deployment}
\newacronym{KB}{KB}{Kilobyte}
\newacronym{MB}{MB}{Megabyte}
\newacronym{NPM}{NPM}{Node Package Manager}
\newacronym{JS}{JS}{Javascript}
\newacronym{GDE}{GDE}{Google Developer Experts}
\newacronym{SEO}{SEO}{Search Engine Optimization}
\newacronym{MVVM}{MVVM}{Model-View-ViewModel}
\newacronym{MVC}{MVC}{Model-View-Controller}
\newacronym{SDK}{SDK}{Software Development Kit}
\newacronym{CPU}{CPU}{Central Processing Unit}
\newacronym{K1}{K1}{Kriterium 1}
\newacronym{K2}{K2}{Kriterium 2}
\newacronym{COP}{COP}{Community of Practice}
\newacronym{TTM}{TTM}{Time-to-Market}
\newacronym{PT}{PT}{Personentage}
\newacronym{PaaS}{PaaS}{Platform as a Service}
\newacronym{VM}{VM}{Virtuelle Maschine}
\newacronym{DB}{DB}{Datenbank}
\newacronym{Lib}{Lib}{Angular Libraries}
\newacronym{B2B}{B2B}{Business-To-Business}
\newacronym{SaaS}{SaaS}{Software as a Service}
\newacronym{TTFB}{TTFB}{Time To First Byte}
\newacronym{TTFD}{TTFD}{Time To First Draw}
\newacronym{DRY}{DRY}{Don't repeat yourself}

% Glossar
\newglossaryentry{CORSHeader}
{
	name=CORS Header,
	description={Durch \textit{Cross-Origin Resource Sharing (CORS)} werden Browsern 
	über \textit{HTTP-Header} mitgeteilt, dass eingebundene Webanwendungen auf anderen 
	Webseiten laufen und diese unter Umständen auf Ressourcen zugreifen können. 
	Die CORS-Header bestimmen, ob und auf welchen Webseiten die konfigurierte 
	Webseite eingebunden werden darf.}
}

\newglossaryentry{Eventbus}
{
	name=Eventbus,
	description={Durch einen \textit{Eventbus} können eine Vielzahl von Empfängern benachrichtigt werden, ohne dass diese sich gegenseitig kennen. Die Empfänger registrieren sich an dem \textit{Eventbus} und reagieren auf Benachrichtigungen, die sie von dort empfangen.}
}

\newglossaryentry{Deeplink}
{
	name={Deep Link},
	description={Ein \textit{Deep Link} ist ein direkter Verweis auf eine (Unter) Seite einer Website. Diese kann Inhalt darstellen oder auch eine Datei sein. Im Gegensatz dazu verweist ein sogenannter \textit{Surface Link} auf die Startseite einer Webseite.}
}

\newglossaryentry{SV}
{
	name=Semantic Versioning,
	description={\textit{Semantic Versioning} ist eine Standardisierung der Versionierung in der Softwareentwicklung. Die Versionszahl besteht aus \textit{Major}, \textit{Minor} und \textit{Patch} \textit{Version}, welche jeweils durch einen Punkt getrennt sind (bspw. v1.2.3). Major Änderungen bewirken Inkompatibilität mit der vorherigen Version. Minor Änderungen sind abwärtskompatibel und eine Patch Änderung bedeutet eine Korrektur zur Stabilität.}
}