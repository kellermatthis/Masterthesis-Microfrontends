%!TEX root = ../Thesis.tex
%% Basierend auf TeXnicCenter-Vorlage von Mark Müller
%%                      Willi Nüßer
%%                      Waldemar Penner     
%%                      Ulrich Reus
%%                      Frank Plass
%%                      Oliver Tribeß 
%%                      Daniel Hintze     
%%%%%%%%%%%%%%%%%%%%%%%%%%%%%%%%%%%%%%%%%%%%%%%%%%%%%%%%%%%%%%%%%%%%%%%

% Wählen Sie die Optionen aus, indem Sie % vor der Option entfernen  
% Dokumentation des KOMA-Script-Packets: scrguide

%%%%%%%%%%%%%%%%%%%%%%%%%%%%%%%%%%%%%%%%%%%%%%%%%%%%%%%%%%%%%%%%%%%%%%%
%% Optionen zum Layout des Artikels                                  %%
%%%%%%%%%%%%%%%%%%%%%%%%%%%%%%%%%%%%%%%%%%%%%%%%%%%%%%%%%%%%%%%%%%%%%%%
\documentclass[%
paper=A4,         % alle weiteren Papierformat einstellbar
fontsize=12pt,    % Schriftgröße (12pt, 11pt (Standard))
BCOR12mm,         % Bindekorrektur, bspw. 1 cm
DIV14,            % breiter Satzspiegel
parskip=half*,    % Absatzformatierung s. scrguide 3.1
headsepline,      % Trennline zum Seitenkopf  
%footsepline,     % Trennline zum Seitenfuß
%normalheadings,  % Überschriften etwas kleiner (smallheadings)
listof=totoc,     % Tabellen & Abbildungsverzeichnis ins Inhaltsverzeichnis      
%bibtotoc,        % Literaturverzeichnis im Inhalt 
%draft            % Überlangen Zeilen in Ausgabe gekennzeichnet
footinclude=false,% Fußzeile in die Satzspiegelberechnung einbeziehen 
headinclude=true, % Kopfzeile in die Satzspiegelberechnung einbeziehen 
final             % draft beschleunigt die Kompilierung
]
{scrartcl}

%\setuptoc{toc}{totoc} % Inhaltsverzeichnis ins Inhaltsverzeichnis

% Neue Deutsche Rechtschreibung und Deutsche Standardtexte
\usepackage[ngerman]{babel} 

% Umlaute können verwendet werden
\usepackage[utf8]{inputenc}   

% Echte Umlaute
\usepackage[T1]{fontenc} 

% Latin Modern Font, Type1-Schriftart für nicht-englische Texte
\usepackage{lmodern} 

% 1/2-zeiliger Zeilenabstand
\usepackage[onehalfspacing]{setspace}

% Für die Defenition eigener Kopf- und Fußzeilen
\usepackage{fancyhdr} 

% Für die Verwendung von Grafiken
\usepackage[pdftex]{graphicx}

% Bessere Tabellen
\usepackage{tabularx}

% Für die Befehle \toprule, \midrule und \bottomrule, z.B. in Tabellen 
\usepackage{booktabs}

% Erlaubt die Benutzung von Farben
\usepackage{color}

% Verbessertes URL-Handling mit \url{http://...}
\usepackage{url}

% Listen ohne Abstände \begin{compactlist}...\end{compactlist}
\usepackage{paralist} 

% Ausgabe der aktuellen Uhrzeit für die Draft-Versionen
\usepackage{datetime}

% Multirow zur Zentrierung von Inhalt in einer Tabellenzelle
\usepackage{multirow}

% to use multi row header
\usepackage{makecell}

% Deutsche Anführungszeichen
\usepackage[babel,german=quotes]{csquotes}

% Konfiguration der Abbildungs- und Tabellenbezeichnungen
\usepackage[format=hang, font={footnotesize, sf}, labelfont=bf, justification=raggedright,singlelinecheck=false]{caption}

% Verbessert die Lesbarkeit durch Mikrotypografie
\usepackage[activate={true,nocompatibility},final,tracking=true,kerning=true,spacing=true,factor=1100,stretch=10,shrink=10]{microtype}  

% Zitate und Quellenverzeichnis
\usepackage[
    bibstyle=authoryear,
    citestyle=authoryear-fhdw,  
    firstinits=false,         % false = Vornamen werden ausgeschrieben
    natbib=true,
    urldate=long,             % "besucht am" - Datum
    %url=false,
    date=long,                
    dashed=false, 
    maxcitenames=3,           % max. Anzahl Autorennamen in Zitaten
    maxbibnames=99,           % max. Anzahl Autorennamen im Quellenverzeichnis
    %backend=bibtex           % Ggf. für ältere Distributionen bibtex verwenden
    backend=biber
]{biblatex}
  
% Bibliograpthy
\bibliography{library/library}

% Keine Einrückung bei einem neuen Absatz 
\parindent 0pt 

% Ebenentiefe der Nummerierung
\setcounter{secnumdepth}{3}

% Gliederungstiefe im Inhaltsverzeichnis 
\setcounter{tocdepth}{3} 

% Tabellen- und Abbildungsverzeichnis mit Bezeichnung:
\usepackage[titles]{tocloft}

% Formeln
\usepackage{amsmath}

% Sourcecode-Listings
\usepackage{listings}

% Bestimmte Warnungen unterdrücken
% siehe http://tex.stackexchange.com/questions/51867/koma-warning-about-toc
\usepackage{scrhack} 

%% http://tex.stackexchange.com/questions/126839/how-to-add-a-colon-after-listing-label
\makeatletter
\begingroup\let\newcounter\@gobble\let\setcounter\@gobbletwo
  \globaldefs\@ne \let\c@loldepth\@ne
  \newlistof{listings}{lol}{\lstlistlistingname}
\endgroup
\let\l@lstlisting\l@listings
\makeatother

\renewcommand*\cftfigpresnum{Abbildung~}
\renewcommand*\cfttabpresnum{Tabelle~}
\renewcommand*\cftlistingspresnum{Listing~}
\renewcommand{\cftfigaftersnum}{:}
\renewcommand{\cfttabaftersnum}{:}
\renewcommand{\cftlistingsaftersnum}{:}
\settowidth{\cftfignumwidth}{\cftfigpresnum 99~\cftfigaftersnum}
\settowidth{\cfttabnumwidth}{\cfttabpresnum 99~\cftfigaftersnum}
\settowidth{\cftlistingsnumwidth}{\cftlistingspresnum 99~\cftfigaftersnum}
\setlength{\cfttabindent}{1.5em}
\setlength{\cftfigindent}{1.5em}
\setlength{\cftlistingsindent}{1.5em}

\renewcommand\lstlistlistingname{Listingverzeichnis}
 
% Style für Kopf- und Fußzeilenfelder
\pagestyle{fancy}
\fancyhf{}
\fancyhead[R]{\leftmark}
\fancyfoot[R]{\thepage} 
\renewcommand{\sectionmark}[1]{\markboth{#1}{#1}} 
\fancypagestyle{plain}{}

% Macro für Quellenangaben unter Abbildungen und Tabellen
\newcommand{\source}[1]{{\vspace{-1mm}\\\footnotesize\textsf{\textbf{Quelle:}} \textsf{#1}\par}}

% Anpassungen der Formatierung an Eclipse-Aussehen 
% http://jevopi.blogspot.de/2010/03/nicely-formatted-listings-in-latex-with.html
%\definecolor{sh_comment}{rgb}{0.12, 0.38, 0.18 } %adjusted, in Eclipse: {0.25, 0.42, 0.30 } = #3F6A4D
%\definecolor{sh_keyword}{rgb}{0.37, 0.08, 0.25}  % #5F1441
%\definecolor{sh_string}{rgb}{0.06, 0.10, 0.98} % #101AF9
% Für Druckausgabe sollte alles schwarz sein
\definecolor{sh_comment}{rgb}{0.0, 0.0, 0.0 }
\definecolor{sh_keyword}{rgb}{0.0, 0.0, 0.0 }
\definecolor{sh_string}{rgb}{0.0, 0.0, 0.0 }

\lstset{ %
  language=Java,
  basicstyle=\small\ttfamily,
  fontadjust, 
  xrightmargin=1mm,
  %xleftmargin=5mm,
  xleftmargin=1mm,
  tabsize=2,
  columns=flexible,
  showstringspaces=false,
  rulesepcolor=\color{black},
  showspaces=false,showtabs=false,tabsize=2,
  stringstyle=\color{sh_string},
  keywordstyle=\color{sh_keyword}\bfseries,
  commentstyle=\color{sh_comment}\itshape,
  captionpos=t,
  lineskip=-0.1em,
  frame = single,
  breaklines=true,
  postbreak=\mbox{\textcolor{red}{$\hookrightarrow$}\space},
}

\lstdefinelanguage{JavaScript}{
	keywords={break, case, catch, continue, debugger, default, delete, do, else, finally, for, function, if, in, instanceof, new, return, switch, this, throw, try, typeof, var, void, while, with},
	morecomment=[l]{//},
	morecomment=[s]{/*}{*/},
	morestring=[b]',
	morestring=[b]",
	sensitive=true
}

%\makeatletter
%\def\l@lstlisting#1#2{\@dottedtocline{1}{0em}{1.5em}{\lstlistingname\space{#1}}{#2}}
%\makeatother

% Anhangsverzeichnis
\usepackage[nohints]{minitoc} %Anhangsverzeichnis

\makeatletter
\newcounter{fktnr}\setcounter{fktnr}{0}
\newcounter{subfktnr}[fktnr]\setcounter{subfktnr}{0}

\renewcommand\thesubfktnr{\arabic{fktnr}.\arabic{subfktnr}}
\newcounter{anhangcounter}
\newcommand{\blatt}{\stepcounter{anhangcounter}}

\newcommand{\anhang}[1]{\setcounter{anhangcounter}{0}\refstepcounter{fktnr}
\addcontentsline{fk}{subsection}{Anhang~\thefktnr: \hspace*{1em}#1}
\subsection*{{Anhang~\thefktnr \hspace*{1em} #1 \hspace*{-1em}}}
}

\newcommand{\subanhang}[1]{\setcounter{anhangcounter}{0}\refstepcounter{subfktnr}
\addcontentsline{fk}{subsubsection}{Anhang~\thesubfktnr: \hspace*{1em}#1}
\subsubsection*{{Anhang~\thesubfktnr \hspace*{1em} #1 \hspace*{-1em}}}
}

\newcommand{\anhangsverzeichnis}{\mtcaddsection{\subsection*{Anhangsverzeichnis \@mkboth{FKT}{FKT}}}\@starttoc{fk}\newpage}

% Links im PDF
\usepackage[pdfpagemode={UseOutlines}, plainpages=false,breaklinks=true,pdfpagelabels]{hyperref}

 % Abkürzungsverzeichnis
\usepackage[acronym,         % create list of acronyms
            nonumberlist,
            toc, 
            section,
            %nomain,          % don't need main glossary for this example
            hyperfirst=false,% don't hyperlink first use
            sanitize=none,    % switch off sanitization as description
            nopostdot
            ]{glossaries}
            \newglossarystyle{mylist}{%
\glossarystyle{long}% base this style on the list style
\renewcommand*{\glossaryentryfield}[5]{%
    \glsentryitem{##1}\textbf{##2} & ##3 \\}%
}

% Verbessert das Referenzieren von Kapiteln, Abbildungen etc.
\usepackage[german,capitalise]{cleveref}
% try to fix Anhang ref but somehow not working...
\crefname{anhang}{Anhang}{Anhänge}
\crefname{subanhang}{Anhang}{Anhänge}
\crefname{ex}{Expertengespräch}{Expertengesprächen}

% Use for reference appendix chapter
\newcounter{ex}
\setcounter{ex}{0}
\newcommand{\deffeat}[1]{
	\phantomsection
	\refstepcounter{ex}
	\label{#1}
	\textbf{Exp~\arabic{ex}}
}

% Akronyme
\newacronym{HTML}{HTML}{HyperText Markup Language}
\newacronym{FHDW}{FHDW}{Fachhochschule der Wirtschaft}
\newacronym{URL}{URL}{Uniform Resource Locator}
\newacronym{CSS}{CSS}{Cascading Style Sheets}
\newacronym{SSI}{SSI}{Server-Side Includes}
\newacronym{DOM}{DOM}{Document Object Model}
\newacronym{API}{API}{Application Programming Interface}
\newacronym{KPI}{KPI}{Key Performance Indicator}
\newacronym{MFE1}{MFE1}{Microfrontend 1}
\newacronym{MFE2}{MFE2}{Microfrontend 2}
\newacronym{MFE3}{MFE3}{Microfrontend 3}
\newacronym{MFE4}{MFE4}{Microfrontend 4}
\newacronym{ID}{ID}{Identifier}
\newacronym{UX}{UX}{User Experience}
\newacronym{UI}{UI}{User Interface}
\newacronym{DDD}{DDD}{Domain-Driven Design}
\newacronym{CDN}{CDN}{Content Delivery Network}
\newacronym{SPA}{SPA}{Single Page Application}
\newacronym{CORS}{CORS}{Cross-Origin Resource Sharing}
\newacronym{OIDC}{OIDC}{OpenID Connect}
\newacronym{SAML}{SAML}{Security Assertion Markup Language}
\newacronym{CI/CD}{CI/CD}{Continuous Integration, Continuous Delivery und Continuous Deployment}
\newacronym{KB}{KB}{Kilobyte}
\newacronym{MB}{MB}{Megabyte}
\newacronym{NPM}{NPM}{Node Package Manager}
\newacronym{JS}{JS}{Javascript}
\newacronym{GDE}{GDE}{Google Developer Experts}
\newacronym{SEO}{SEO}{Search Engine Optimization}
\newacronym{MVVM}{MVVM}{Model-View-ViewModel}
\newacronym{MVC}{MVC}{Model-View-Controller}
\newacronym{SDK}{SDK}{Software Development Kit}
\newacronym{CPU}{CPU}{Central Processing Unit}
\newacronym{K1}{K1}{Kriterium 1}
\newacronym{K2}{K2}{Kriterium 2}
\newacronym{COP}{COP}{Community of Practice}
\newacronym{TTM}{TTM}{Time-to-Market}
\newacronym{PT}{PT}{Personentage}
\newacronym{PaaS}{PaaS}{Platform as a Service}
\newacronym{VM}{VM}{Virtuelle Maschine}
\newacronym{DB}{DB}{Datenbank}
\newacronym{Lib}{Lib}{Angular Libraries}
\newacronym{B2B}{B2B}{Business-To-Business}
\newacronym{SaaS}{SaaS}{Software as a Service}
\newacronym{TTFB}{TTFB}{Time To First Byte}
\newacronym{TTFD}{TTFD}{Time To First Draw}
\newacronym{DRY}{DRY}{Don't repeat yourself}

% Glossar
\newglossaryentry{CORSHeader}
{
	name=CORS Header,
	description={Durch \textit{Cross-Origin Resource Sharing (CORS)} werden Browsern 
	über \textit{HTTP-Header} mitgeteilt, dass eingebundene Webanwendungen auf anderen 
	Webseiten laufen und diese unter Umständen auf Ressourcen zugreifen können. 
	Die CORS-Header bestimmen, ob und auf welchen Webseiten die konfigurierte 
	Webseite eingebunden werden darf.}
}

\newglossaryentry{Eventbus}
{
	name=Eventbus,
	description={Durch einen \textit{Eventbus} können eine Vielzahl von Empfängern benachrichtigt werden, ohne dass diese sich gegenseitig kennen. Die Empfänger registrieren sich an dem \textit{Eventbus} und reagieren auf Benachrichtigungen, die sie von dort empfangen.}
}

\newglossaryentry{Deeplink}
{
	name={Deep Link},
	description={Ein \textit{Deep Link} ist ein direkter Verweis auf eine (Unter) Seite einer Website. Diese kann Inhalt darstellen oder auch eine Datei sein. Im Gegensatz dazu verweist ein sogenannter \textit{Surface Link} auf die Startseite einer Webseite.}
}

\newglossaryentry{SV}
{
	name=Semantic Versioning,
	description={\textit{Semantic Versioning} ist eine Standardisierung der Versionierung in der Softwareentwicklung. Die Versionszahl besteht aus \textit{Major}, \textit{Minor} und \textit{Patch} \textit{Version}, welche jeweils durch einen Punkt getrennt sind (bspw. v1.2.3). Major Änderungen bewirken Inkompatibilität mit der vorherigen Version. Minor Änderungen sind abwärtskompatibel und eine Patch Änderung bedeutet eine Korrektur zur Stabilität.}
}
\makeglossaries\makeglossaries 

% Command for counter equations
\newcommand\numberthis{\addtocounter{equation}{1}\tag{\theequation}} 

% cleveref Names
\addto\captionsngerman{
	% Second argument is singular, third is plural
	\crefname{subsection}{Abschnitt}{Abschnitten}
	\crefname{section}{Kapitel}{Kapitel}
}