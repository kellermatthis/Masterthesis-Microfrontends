%!TEX root = ../Thesis.tex
\section{Einleitung}\label{ch:Einleitung}

In der nachfolgenden Masterthesis zum Thema \textit{\dokumententitel} wird das Konzept einer Microfrontend-Architektur anhand einer Portalshell und mehreren eingebundenen Microfrontends erklärt.

\subsection{Motivation}\label{sec:Aufgabenstellung}

Microservices sind seit vielen Jahren in der Softwareentwicklung etabliert. Mittlerweile sind zwei Drittel der repräsentativen Unternehmen auf dem Weg, ihre Software basierend auf Microservice-Technologien zu cloud-native Lösungen umzubauen.\footnote{\cite[vgl.][]{Lünendonk2021}}

Ein vergleichbarer Trend ist im Bereich von Frontends zu beobachten. In den vergangenen Jahren sind Microfrontends vermehrt auf dem Technologieradar erschienen.\footnote{\cite[vgl.][]{Thoughtworks2020}} Microfrontends bieten ebenso wie die Microservices viel Potential, sind aber noch weniger verbreitet. Durch Microfrontend-Architekturen können komplexe Webapplikationen realisiert werden, welche zeitgleich skalierbar sowie flexibel sind und von autonomen, vertikalen Teams betreut werden. Welche Microfrontend-Architekturen für den jeweiligen Anwendungsfall in Frage kommen und welche Besonderheiten diese jeweils mit sich bringen ist aufgrund der großen Entscheidungsvielfalt nicht immer direkt ersichtlich.

Zum aktuellen Zeitpunkt planen 73\% der Firmen alle ihre Softwaresysteme auf \gls{SaaS}-Lösungen umzustellen.\footnote{\cite[vgl.][]{Alves2021}} Unternehmen, welche im \gls{B2B} Geschäftsfeld tätig sind, können von diesem Trend profitieren und umfangreiche, aber zeitgleich auf den individuellen Anforderungen basierende Softwarelösungen für ganze Branchen anbieten.\newline
Dafür würde beispielsweise die Erstellung einer mandantenfähigen Portalapplikation in Frage kommen, in welcher \gls{B2B}-Kunden ihre benötigten Tools in Form von Microfrontends individuell buchen, konfigurieren und nutzen können.

\subsection{Zielsetzung}\label{sec:Zielsetzung}

Ziel dieser \dokumententyp{} ist es, eine Entscheidungsgrundlage für Portalapplikationen zu schaffen, anhand derer die optimale Art der Einbindung für Microfrontends gefunden werden kann. 
Die Entscheidungsfindung muss aufgrund der verschiedenen Handlungsoptionen fundiert durchgeführt werden, damit die Ergebnisse anwendbar sind. 
Jedes Microfrontend soll optimal in die Portalapplikation integriert sein, auf Basis der individuellen Anforderungen der Portalapplikation, des Nutzers und des jeweiligen Microfrontends.

\subsection{Vorgehensweise}\label{sec:Vorgehensweise}

Um das Ziel zu erreichen gliedert sich die \dokumententyp{} in sechs Kapitel. Nach der Einleitung in \cref{ch:Einleitung}, welche Motivation, Zielsetzung und die Vorgehensweise beschreibt, werden in \cref{sec:Grundlagen} die Grundlagen für die Arbeit gelegt. Die Grundlagen gehen auf Microservices, Microfrontends und Portalapplikationen ein, welche den fachlichen Rahmen der Arbeit ausmachen.

Anschließend wird in \cref{sec:PrototypischesBeispiel} das prototypische Beispiel skizziert, anhand dessen in \cref{sec:Evaluierung} die Evaluierung durchgeführt wird. Die Evaluierung beinhaltet zunächst die Erläuterung der theoretischen Grundlagen einer Nutzwertanalyse, welche dann anschließend in den \crefrange{sec:KriterienArtenEinbindung}{sec:VergleichDerArten} anhand des prototypischen Beispieles durchgeführt wird. 

Im letzten Abschnitt des vierten Kapitels (\ref{sec:OptimaleAnwendungsszenarien}) wird das Ergebnis der Nutzwertanalyse interpretiert und aufbereitet. In diesem Abschnitt wird ein Prozess definiert, welcher einen Leitfaden bei der Entscheidungsfindung zur Einbindung von Microfrontends in eine Portalshell bilden soll.

Das vorletzte Kapitel beinhaltet die prototypische Implementierung anhand der vorherigen Erkenntnisse. Dort soll der entwickelte Prozess für einige exemplarische Microfrontends angewendet werden. Anschließend werden die Ergebnisse bewertet.

Das letzte Kapitel enthält eine Schlussbetrachtung über die vorangegangene \dokumententyp{}. Im Zuge dieser wird die Erfüllung der festgelegten Zielsetzung überprüft und eine Wertung vollzogen. Anschließend wird ein Ausblick gegeben, welche weiteren Aspekte rund um das ausgewählte Thema ebenfalls untersucht werden könnten.