%!TEX root = ../Thesis.tex
\subsection{Aufbau und Struktur der Arbeit}

In Kapitel \cref{sec:installation} wird die Installation von MikTex, dem TeXnicCenter und einiger Tools erklärt. (Anmerkung: MikTeX enthält ab Version 2.8 den Editor TeXworks.) Die Ausführungen beziehen sich hauptsächlich auf Programme für das Betriebssystem ``Microsoft Windows''. Unter Linux bzw. Mac-OS gibt es aber ähnliche Distributionen bzw. Editoren. Wo bekannt, wird auf Unterschiede hingewiesen.

Im Anschluss werden die verwendeten Pakete vorgestellt und erläutert. Es wird nicht auf grundsätzliches zu Latex eingegangen. Dieses kann in entsprechenden Dokumenten nachgelesen werden.\footnote{\cite[siehe][]{einf:latex,fort:latex}}